\newcommand\conciq{
\npara
Our aim in this section is to prove following theorem:
\begin{theorem}
There exist absolute constant $C_{CW}$ such that if $f:\IC^n \to \IC$ is homogeneous polynomial of degree $r$, $\|f\|_2 = 1$ and $\delta>0$ be such that $1/\delta$ is an integer then for any $\alpha > 0$, 
$$
\Pr_{\BF v\in G_{\delta}^{\IC}}\Big[|f(\BF v)|\le \alpha - \frac{1}{2}\delta(16nr^2)^{2n+1}\Big]\le C_{CW}\cdot r\cdot(2\alpha)^{1/r}.
$$
\end{theorem}
In that direction, we first prove continuous version\footnote{You will understand why we call it "continuous version".} of the theorem on reals, then we prove discrete version on reals using already proved continuous version and finaly we give the proof of above theorem. Before starting we state few definations and facts required in this section.

\begin{definition}[log-concave probability measure]
Probability measure\footnote{Probability measure is just usual measure with $\mu(\Omega)=1$.} $\mu$ on probability space $(\IR^n,\SB,\mu)$ is log-concave if function $(\log\circ\mu)$ is concave. That is, 
$$
\mu(\lambda A + (1-\lambda)B) \ge \mu(A)^{\lambda}\mu(B)^{1-\lambda}
$$	
where $\SB$ is $\sigma$-algebra\footnote{$\sigma$-Algebra $\SB\subset\POWSET(\IR^n)$ is set which is closed under complementation, countable union and $\emptyset\in\SB$.} generated by usual topology on $\IR^n$; $A,B\in\SB$ and $\lambda \in (0,1)$.
\end{definition}
\begin{fact}
Uniform probability measure over convex set is log concave.
\end{fact}
\thmpara
\begin{proof}
\end{proof}
We use following result [CW01] without proving it here.
\begin{theorem}[{[CW01]}]
There exist an absolute constant $C$ such that if $f:\IR^n \to \IR$ is a polynomial of degree at most $r$, $0<q<\infty$, $\mu$ is log-concave probability measure on $\IR^n$ then
$$
\Bigg(\int |f(v)|^{q/r}d\mu(x)\Bigg)^{1/q}\cdot \mu(\{v:|f(v)|\le\alpha\}) \le C\cdot q\cdot \alpha^{1/r}
$$
for any $\alpha \ge 0$.
\end{theorem}
Following is the continuous version of Theorem 1 on reals.
\begin{corollary}[]
There exist an absolute constant $C_{CW}$ such that if $f:\IR^n \to \IR$ is a polynomial of degree $r$ and $\|f\|_2 = 1$, then for any $\alpha > 0$
$$
\Pr\Big[|f(v)| \le \alpha\Big] \le C_{CW}\cdot r\cdot\alpha^{1/r}
$$
\end{corollary}
\begin{proof}
Apply Theorem (..) for uniform measure on $[-1,1]^n$ and $q=2r$,
$$
\int_{[-1,1)^n}|f(\BF x)|^{q/r}d\mu(\BF x)=\int_{[-1,1)^n}|f(x)|^2d\mu(x)=\|f\|^2_2=1.
$$
Now the claim is trivial.
\end{proof}
\thmpara
Let $\delta > 0$ such that $1/\delta$ is an integer and $G_{\delta} = \{-1,-1+\delta,-1+2\delta,...,1-2\delta,1-\delta\}^{n} \subset [-1,1)^{n}$. For $v \in [-1,1)^{n}$ we define $\UL v$ as $\UL v_i = m_i\delta$ where $m_i$ is an integer such that $m_i\delta \le v_i < (m_i + 1)\delta$.

\begin{lemma}
Let $f:\IR^n \to \IR$ be the polynomial function such that $\|f\|_2=1$, $\delta>0$ such that $1/\delta$ is an integer then $|f(\BF v)-f(\UL{\BF{v}})|\le \delta \cdot (8nr^2)^{n+1}$.
\end{lemma}
\begin{proof}
By mean value theorem there exist point $\BF{w}$ on line segment joining $\BF v$ and $\UL{\BF v}$ and if $f'(\BF w)$ denote derivative of $f$ at $\BF w$ in the direction of $v-\UL v$ then
\begin{align*}
|f(\BF v) - f(\BF{\UL v})|&=\|\BF v - \BF{\UL v}\|\cdot f'(\BF w) \\
&\le \|\BF v-\BF{\UL v}\|\cdot 2r^2 \cdot \|f\|_\infty\tag{By Multivariate Markov Inequality.}\\
&\le \|\BF v- \BF{\UL v}\| \cdot 2r^2 \cdot \|f\|_2 \cdot 2^{n+1}\cdot \frac{1}{\big(vol(n,\frac{1}{4r^2})\big)^{1/2}}\tag{By Lemma (..)}
\end{align*}
We have that $\|\BF v - \BF{\UL v}\| \le \delta\sqrt{n}$ and since cube of side $\sqrt{2}R$ can be inscribed in ball of radius $R$ $vol(n,\frac{1}{4r^2}) \ge \Big(\frac{1}{2\sqrt 2 r^2}\Big)^n\ge\Big(\frac{1}{2\sqrt 2 nr^2}\Big)^n$.
\begin{align*}
|f(\BF v)-f(\BF{\UL v})|&\le \delta\sqrt{n}\cdot2r^2 \cdot \|f\|_2 \cdot 2^{n+1}\cdot (2\sqrt{2}nr^2)^{n/2}\le \delta\cdot(8nr^2)^{n+1}\cdot\|f\|_2.
\end{align*}
\end{proof}
\begin{corollary}
Let $f:\IR^n\to\IR$ be such that $\|f\|_2=1$, $\delta>0$ such that $1/\delta$ is an integer and $\BF v \in [-1,1]^n$ such that $|f(\BF v)| > \alpha$ for some $\alpha>0$. Then $|f(\BF{\UL v})| > \alpha - \delta\cdot(8nr^2)^{n+1}$.
\end{corollary}
\begin{proof}
$$
\alpha < |f(\BF v)| \le |f(\BF v)-f(\BF{\UL v})|+|f(\BF{\UL v})|\le\delta\cdot(8nr^2)^{n+1} + |f(\BF{\UL v})|
$$
\end{proof}
\begin{theorem}
Let $f:\IR^n \to \IR$ be the polynomial function such that $\|f\|_2=1$, $\delta>0$ such that $1/\delta$ is an integer then for any $\alpha>0$ 
$$
\Pr_{\BF{\UL v} \in_{u} G_{\delta}}\Big[|f(\BF{\UL v})| \le \alpha-\delta(8nr^2)^{n+1}\Big] \le C_{CW}\cdot r \cdot \alpha^{1/r}
$$
Where $C_{CW}$ is the constant guaranteed in Corollary 5. 
\end{theorem}
\begin{proof}
Let $\BF v\in_{\UNI} [-1,1]^n$, by Corollary 5, 
$$
\Pr\Big[\big|f(\BF v)\big| > \alpha\Big] > 1-C_{CW}\cdot r \cdot \alpha^{1/r}.
$$
By corollary 7,
$$
\Pr\Big[\big|f(\BF{\UL v})\big| > \alpha-\delta(8nr^2)^{n+1}\ given\ \big|f(\BF v)\big| > \alpha\Big] = 1.
$$
Thus we have,
$$
\Pr\Big[\big|f(\BF{\UL v})\big| > \alpha-\delta(8nr^2)^{n+1}\Big] = 1-C_{CW}\cdot r \cdot \alpha^{1/r}.
$$.
$$
\therefore \Pr_{\BF{\UL v} \in_{u} G_{\delta}}\Big[|f(\BF{\UL v})| \le \alpha-\delta(8nr^2)^{n+1}\Big] \le C_{CW}\cdot r \cdot \alpha^{1/r}.
$$
\end{proof}
Next we give proof of Theorem 1. Let $f:\IC^n\to \IC$ be a polynomial of degree $r$. $\Re(f),\Im(f):\IR^{2n}\to \IR$ be the real and imaginary parts of $f$, that is $f(\BF a + \iota\BF b) = \Re(f)(\BF a,\BF b)+\iota\Im(f)(\BF a,\BF b)$. Note that $\Re(f)$ and $\Im(f)$ are polynomials of degree $r$. We define $\|f\|_2 \DEF \|\Re(f)\|_2+\|\Im(f)\|_2$.
\begin{proof}[Proof of Theorem 1]
Since $\|f\|_2 = 1$, either $\|\Re(f)\|_2\ge 1/2$ or $\|\Im(f)\|_2 \ge 1/2$. Assume w.l.o.g $\|\Re(f)\|\ge 1/2$. By applying Theorem 8 to $\Re(f)$ and $\alpha'$, 
$$
\Pr\Big[|\Re(f)(\BF a, \BF b)| > \big(\alpha' - \delta \cdot (8(2n)r^2)^{2n+1}\big)\cdot\|\Re(f)\|_2\Big] > 1-C_{CW}\cdot r \cdot (\alpha')^{1/r}.
$$
Note that, for any $\gamma$, $|\Re(f)(\BF a,\BF b)|>\gamma$ then $|f(\BF a + \iota \BF b)|>\gamma$. Using $\|\Re(f)\|_2\ge 1/2$ and $\alpha'=2\alpha$ we have, 
$$
\Pr\Big[|f(\BF a+\iota\BF b)| > \big(2\alpha - \delta \cdot (16nr^2)^{2n+1}\big)\cdot \frac{1}{2}\Big] > 1-C_{CW}\cdot r \cdot (2\alpha)^{1/r}.
$$
$$
\therefore\Pr\Big[|f(\BF a+\iota\BF b)| \le \big(\alpha - \frac{1}{2}\cdot  \delta \cdot(16nr^2)^{2n+1}\big)\Big] \le C_{CW}\cdot r \cdot \alpha^{1/r}.
$$

\end{proof}
}
