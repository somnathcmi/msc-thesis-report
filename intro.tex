\newcommand\intro{
\npara
\section{Introduction}
In the paper [FS17] authors study the following problem. What is the complexity of constructing a set that is guaranteed to be a hitting set for \OL{VP}? Recall that $\SH$ is a hitting set for a class of polynomials $\SC$ if for every $f\in\SC$ there is some $v\in\SH$ such that $f(\VEC v)\ne0$. \OL{VP} consist of polynomials that can be infinitesimally approximated by small algebraic circuits.

The question of constructing a hitting set for \OL{VP} that is guaranteed to work was raised by Mulmuley [Mul17]. Specifically, Mulmuley asked what is the complexity of constructing a set that is guaranteed to be a hitting set for VP 
and \OL{VP}. For VP, he proved that it can be done in PSPACE using the results in [HS80a], [Raz10] and [Koi96]. In the paper authors solve the problem for \OL{VP}.

\subsection{Sketch of proof}

First we will study the PSPACE algorithms which construct hitting set for VP. The idea is that one can enumerate over all subsets of search space and for each such subset check whether there exists a circuit that computes a nonzero polynomial that vanishes over the subset. To get a PSPACE algorithm from the idea, first we need to prove the existence of search space which can be enumerated in PSPACE, this is already done in [HS80a], the paper also proves that size of hitting set is polynomially bounded. Second we need a procedure which can check in PSPACE that the subset is hitting set or not, this can be done using universal circuit. The universal circuit $\Psi(\VEC x,\VEC y)$ is a circuit in $n$ essential variables $\VEC x$ and $poly(r,s)$ auxiliary variables $\VEC y$ such that for any size $s$ and degree $r$ circuit $\phi(\VEC x)$ there is an assignment $\VEC a$, to the auxiliary variables, so that the polynomials computed by $\Psi(\VEC x,\VEC a)$ and $\phi(\VEC x)$ are the same. Thus, if our subset is $\VEC v_1,...,\VEC v_m$, where $m$ is polynomially bounded and if we can check whether there is $\VEC y$ such that 
$$\forall i\big(\Psi(\VEC v_i,\VEC y)=0\big)\ {and}\ \exists\VEC u\big(\psi(\VEC u,\VEC y)=1\big).$$
then $\VEC v_1,...,\VEC v_m$ is not a hitting set. Above expression can be checked in PSPACE using Hilbert's Nullstellensatz. The problem of deciding whether a system of polynomial equalities has a complex solution is known as Hilbert's Nullstellensatz problem in the computer science literature and it is solvable in PSPACE, see [Koi96].

We would like to use similar approach to construct hitting set for \OL{VP}. The problem is that even if $\SH$ is a hitting set for the $n$-variate circuits of size $s$ and degree $r$, it may be the case that for a sequence of polynomials $\{f_i\}$, even if $f_i(\VEC v)\ne 0$ for all $i$, the limit polynomial may still vanish at $\VEC v$. Thus, it is not clear that $\SH$ also hits the closure of $n$ variate circuits of size $s$ and degree $r$.

To overcome the discrepancy between a hitting set for VP and a hitting set for VP, we would like to find what we call a "robust hitting set", a set of inputs is said to be a $\eta$-robust hitting set if for every nonzero polynomial that can be computed by a polynomial sized algebraic circuit there is an element of the set on which this polynomial evaluates to at least $\eta>0$, after adequate normalization. Thus, if $f_i$ are all normalized and evaluate to at least $\eta$ on $\VEC v\in\SH$, then if $\lim f_i=f$ then by continuity $f$ also evaluates to at least $\eta$ on $\VEC v$. Thus, $\SH$ hits $f$ as well.

Therefore, the first step in our proof is to prove the existence of poly size robust hitting set and a search space which can be enumerated in PSPACE. We need to move to $\IC$ since we want to use few results in algebraic geometry. Then our search space for robust hitting set will be $$G_\delta^\IC=\{\VEC a+\iota\VEC b:\VEC a,\VEC b\in \{-1,-1+\delta,-1+2\delta,...,1-2\delta,1-\delta\}^n\}.$$ We note that [HS80a] proved the existence of a poly size hitting set for $n$-variate circuits of size $s$ and degree $r$, but their proof does not yield robust hitting set. To prove the existence of robust hitting set which is subset of $G_\delta^\IC$, we think of a polynomials as points represented by coefficient vectors in euclidean space. we use the bounds given by Heintz and Sieveking [HS80b] on the dimension, denoted by $d$, and degree, denoted by $D$, of the algebraic variety of efficiently computable polynomials, we denote the algebraic variety by $V(n,s,r)$. Following are the steps to prove existence of robust hitting set:

\begin{itemize}
\item Prove existence of $\epsilon$-net $E$ for $V(n,s,r)$ and find the bound on size of $E$. The proof we give is geometric in nature and bound we will get on $|E|$ is $D\cdot(\frac{N}{\epsilon})^{\bigo{d}}$ where $N$ is number of $n$-variate monomials of degree $r$.
\item For some $\eta>0$, prove existence of poly size $\eta$-robust hitting set $\SH$ for $E$. The proof is probabilistic in nature and uses anti-concentration result for polynomials proved in [CW01].
\item Prove if $\SH$ is $\eta$-robust hitting set for $E$ then for some $\eta'>0$ it is $\eta'$-robust hitting set for $V(n,s,r)$.
The proof uses few results related to norms of polynomial and find the relation between $\eta$ and $\eta'$.
\end{itemize}
Now that we know that robust hitting sets exist the PSPACE algorithm works as follows. It enumerates over all subsets of a $G_\delta^\IC$ of polynomial size. For each such subset it checks whether there exists an normalized algebraic circuit that evaluates to at most $\eta$ on all points in the subset. If such a solution is found then the subset is not robust and we move to the next subset. To check whether such a solution exists we need to express this system of inequalities as a formula in the language of the existential theory of the reals. Then we use the fact that formulas in this language can be decided in PSPACE to conclude that our algorithm works in PSPACE.
}